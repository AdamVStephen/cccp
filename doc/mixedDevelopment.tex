
# TODO : how to use cite in LaTeX

\section{Introduction}

Document conventions

\begin{enumerate}
\item a.out : generically the output file from a compiler in example codes.
\end{enumerate}

\section{Building with CMake}

\subsection{C++ Example}

\cite{https://cmake.org/cmake-tutorial/}

\section{Exploring a Binary}

\cite{https://unix.stackexchange.com/questions/719/can-we-get-compiler-information-from-an-elf-binary}

Tips from this article include :

\begin{enumerate}
\item No requirement to include compiler invocation metadata in ELF but optional.  GCC uses .comment sections.
\item readelf, objdump and strings are all useful tools to interrogate the ELF data.
\end{enumerate}

\cite{https://jvns.ca/blog/2014/09/06/how-to-read-an-executable/}

Tips from this article include :

\begin{enumerate}
\item .interp section from objdump -s lists the path of the dynamic linker

\item sections are used a link time by ld
\item segments are used at execution time (.text, .data, .rodata)
\item .text segment contains assembly code
\item Links to introduction to ELF \cite{http://www.bottomupcs.com/elf.html}
\item Links to graphic showing ELF structure \cite{https://code.google.com/p/corkami/wiki/ELF101}
\item Links to 20 part series on linkers \cite{https://lwn.net/Articles/276782/}
\item Links to Editing Binaries (easier than you may think) \cite{https://danluu.com/edit-binary/}
\end{enumerate}


\subsection{Techniques}

\begin{enumerate}
\item objdump -s --section .comment a.out
\item readelf -p .comment a.out
\item strings -a a.out | grep gcc
\item objdump -sj .note.ABI-tag ELFFILE
\item objdump -sj .note.gnu-build-id ELFFILE
\item readelf -n ELFFILE # -n notes
\item objcopy # allows not only copy, but annotate by adding more information.
\item readelf --symbols # equivalent of nm
\item objdump -tT /path/to/libc.so # will tell you about symbols of a shared lib when nm cannot.
\item readelf --sections
\item objdump -d # Disassemble the contents of a .text segment to read the code.
\item readelf --segments # segments determine where in memory code/data get loaded and how they are used/accessed
\end{enumerate}

\section{Examples}

\begin{quote}
$>objdump -s --section .comment ./Tutorial 

./Tutorial:     file format elf64-x86-64

Contents of section .comment:
 0000 4743433a 20285562 756e7475 20382e33  GCC: (Ubuntu 8.3
 0010 2e302d31 36756275 6e747533 7e31362e  .0-16ubuntu3~16.
 0020 30342920 382e332e 3000               04) 8.3.0.      
\end{quote}


4


You can also use a clever script (analyze-x86_64.v2.sh) that counts the numbers of various CPU instructions used by the binary. It is based on parsing objdump output. Beware that it can take quite a long time to finish if you use it on a big binary.

https://pastebin.com/EU23iAvL

\section{To Research}

\begin{enumerate}
\item How to intercept and/or modify \_start which calls main
\end{enumerate}

\section{On Python}

https://jvns.ca/blog/2014/08/12/what-happens-if-you-write-a-tcp-stack-in-python/

\section{References}

[1] https://unix.stackexchange.com/questions/719/can-we-get-compiler-information-from-an-elf-binary
